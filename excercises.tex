\documentclass{scrartcl}

\usepackage[a4paper]{geometry}

\usepackage{xecyr}
\usepackage{polyglossia}
\setmainlanguage{bulgarian}

\usepackage{libertineotf}

\usepackage{amsmath}
\newtheorem{pr}{Задача}[section]
\newtheorem{q}{Въпрос}[section]
\newcommand\ans{Отг.: }
\newcommand\rans{\hfill\ans}

\renewcommand{\theenumi}{\alph{enumi}}

\newenvironment{itemize*}%
  {\begin{itemize}%
    \setlength{\itemsep}{0pt}%
    \setlength{\parskip}{0pt}}%
  {\end{itemize}}

\newenvironment{defractors}{
\begin{enumerate}
  \setlength{\itemsep}{1pt}
  \setlength{\parskip}{0pt}
  \setlength{\parsep}{0pt}
}{\end{enumerate}}

\usepackage{hyperref}
\usepackage{booktabs}

\begin{document}

\title{Упражнения}
\subtitle{избрани въпроси по компютърни мрежи и комуникации}
\author{\href{mailto:bordjukov@gmail.com}{П. Борджуков}}
\date{}

\maketitle

%%% Local Variables:
%%% TeX-master: "excercises"
%%% ispell-check-comments: exclusive
%%% ispell-local-dictionary: "bg"
%%% End:

\section{Увод в TCP/IP}
\subsection{Въпроси относно OSI модела}

\begin{q}
  Кой слой е отговорен за конвертирането на данните от каналния слой в
  електрически импулси?
\end{q}

\begin{q}
  В кой слой е имплементирано маршрутизирането, позволяващо свързването и
  избирането на път за пренос на данни между две крайни системи?
\end{q}

\begin{q}
  Кой слой определя как се форматират, представят, кодират и конвертират
  мрежовите данни?
\end{q}

\begin{q}
  Кой слой е отговорен за създаването, управляването и прекратяването на сесии
  между приложения?
\end{q}

\begin{q}
  Кой слой осигурява сигурното предаване на данни през физическата среда и
  отговаря основно за физическото адресиране, дисциплината на линията, мрежовата
  топология, нотификацията за грешки, преноса на рамки в правилен ред и контрола
  на потока?
\end{q}

\begin{q}
  Кой слой се използва за надеждна комуникация между крайни хостове в мрежата и
  предоставя механизми за установяване, поддържане и прекратяване на виртуални
  вериги, откриване на и възстановяване от грешки, възникнали при транспорта на
  данни, и контрол на потока на информация?
\end{q}

\begin{q}
  Кой слой предоставя логическо адресиране, което маршрутизаторите използват за
  установяване на маршрут за пренос на данни?
\end{q}

\begin{q}
  Кой слой определя волтажа, скоростта и изводите (pinout) на проводника и
  предава битове между мрежови устройства?
\end{q}

\begin{q}
  Кой слой комбинира битове в байтове и байтове в рамки, използва MAC адресиране
  и установява дали са възникнали грешки по време на преноса на данните във
  физическата среда?
\end{q}

\begin{q}
  Кой слой е отговорен за разграничаването на данните от различните приложения
  (мултиплексиране) при мрежова комуникация?
\end{q}

\begin{q}
  Продукт на кой слой са рамките?
\end{q}

\begin{q}
  Продукт на кой слой са сегментите?
\end{q}

\begin{q}
  Продукт на кой слой са пакетите?
\end{q}

\begin{q}
  Продукт на кой слой са битовете?
\end{q}

\begin{q}
  Поставете следните единици данни в ред на енкапсулация, започвайки от
  най-вътрешната:

  \begin{itemize}
  \item Пакети
  \item Рамки
  \item Битове
  \item Сегменти
  \end{itemize}
\end{q}

\begin{q}
  Кой слой сегментира и реасемблира данните?
\end{q}

\begin{q}
  Кой слой се грижи за привеждането на данните във формат, удобен за предаване
  на физическо ниво и отговаря за нотификацията при възникване на грешки,
  мрежовата топология и контрола на потока?
\end{q}

\begin{q}
  Кой слой управлява адресирането на устройствата, проследява положението на
  устройствата в мрежата и определя най-добрия път за пренос на данни?
\end{q}

\begin{q}
  Каква е дължината в битове и в какъв вид се изразява MAC адресът?
\end{q}
\section{IP адресация}
\begin{q}
  Напишете адреса на подмрежата, broadcast адреса на подмрежата и интервала от
  валидни адреси на хостове за всяка от следните двойки адрес и маска:
  \begin{enumerate}
  \item \texttt{192.168.100.25/30}
  \item \texttt{192.168.100.37/255.255.255.240}
  \item \texttt{192.168.100.66/255.255.255.224}
  \item \texttt{192.168.100.17/29}
  \item \texttt{192.168.100.99/26}
  \item \texttt{192.168.100.99/255.255.255.128}
  \end{enumerate}
\end{q}

\begin{q}
  Имате клас B мрежа и се нуждаете от 29 подмрежи. Каква мрежова маска ще
  изберете? \rans \texttt{255.255.248.0} или \texttt{/21}
\end{q}

\begin{q}
  Какъв е broadcast адресът на подмрежата, в която се намира хостът с адрес
  \texttt{192.168.192.10/29}? \rans \texttt{192.168.192.15}
\end{q}

\begin{q}
  Колко адреса за хостове предлага подмрежа с маска \texttt{/29}? \rans 6
\end{q}

\begin{q}
  Какъв е адресът на подмрежата на \texttt{10.16.3.65/23}? \rans
  \texttt{10.16.2.0}
\end{q}

\begin{q}
  Попълнете следната таблица:

  \begin{center}
    \begin{tabular}{ r c c } \toprule
      \textbf{\emph{CIDR}} & \textbf{\emph{Маска на подмрежа (dot-decimal)}} &
        \textbf{\emph{Брой хостове в подмрежа}} \\ \midrule
      /16 & & \\ \hline
      /17 & & \\ \hline
      /18 & & \\ \hline
      /19 & & \\ \hline
      /20 & & \\ \hline
      /21 & & \\ \hline
      /22 & & \\ \hline
      /23 & & \\ \hline
      /24 & & \\ \hline
      /25 & & \\ \hline
      /26 & & \\ \hline
      /27 & & \\ \hline
      /28 & & \\ \hline
      /29 & & \\ \hline
      /30 & & \\ \bottomrule
    \end{tabular}
  \end{center}
\end{q}

\begin{q}
  Попълнете следната таблица:

  \begin{center}
    \begin{tabular}{@{}lc p{2cm} p{2cm} c c@{}} \toprule
      & & \multicolumn{2}{c}{Брой битове за} & \multicolumn{2}{c}{Брой} \\
      Адрес & Клас & \centering подмрежа & \centering хост & подмрежи & хостове
      \\ \midrule
      \texttt{10.25.66.154/23} & & & & & \\
      \texttt{172.31.254.12/24} & & & & & \\
      \texttt{192.168.20.123/28} & & & & & \\
      \texttt{63.24.89.21/18} & & & & & \\
      \texttt{128.1.1.254/20} & & & & & \\
      \texttt{208.100.54.209/30} & & & & \\ \bottomrule
    \end{tabular}
  \end{center}
\end{q}

\begin{q}
  Какъв е максималният брой IP адреси, които могат да бъдат зачислени на хостове
  в локална подмрежа с маска \texttt{255.255.255.224}?
  \begin{enumerate}
  \item 14
  \item 15
  \item 16
  \item 30
  \item 31
  \item 62
  \end{enumerate}
\end{q}

\begin{q}
  Имате мрежа, която трябва да разделите на 29 подмрежи, предлагащи възможно
  най-голям брой адреси на хостове. Колко бита трябва да заемете от полето на
  хоста, за да постигнете това?
  \begin{enumerate}
  \item 2
  \item 3
  \item 4
  \item 5
  \item 6
  \item 7
  \end{enumerate}
\end{q}

\begin{q}
  Имате хост с IP адрес \texttt{200.10.5.68/28}. Кой е адресът на подмрежата, от
  която е част този хост?
  \begin{enumerate}
  \item \texttt{200.10.5.56}
  \item \texttt{200.10.5.32}
  \item \texttt{200.10.5.64}
  \item \texttt{200.10.5.0}
  \end{enumerate}
\end{q}

\begin{q}
  Колко подмрежи и колко адреса за хостове в подмрежа предоставя мрежовият адрес
  \texttt{172.16.0.0/19}?
  \begin{enumerate}
  \item $7$ подмрежи, $30$ хоста във всяка
  \item $7$ подмрежи, $2046$ хоста във всяка
  \item $7$ подмрежи, $8190$ хоста във всяка
  \item $8$ подмрежи, $30$ хоста във всяка
  \item $8$ подмрежи, $2046$ хоста във всяка
  \item $8$ подмрежи, $8190$ хоста във всяка
  \end{enumerate}
\end{q}

\begin{q}
  Кои две твърдения са верни за IP адреса \texttt{10.16.3.65/23}?
  \begin{enumerate}
  \item Адресът на подмрежата му е \texttt{10.16.3.0/255.255.254.0}.
  \item Най-ниският адрес на хост в подмрежата му е \texttt{10.16.2.1}.
  \item Последният валиден адрес на хост в подмрежата му е \texttt{10.16.2.254}.
  \item Broadcast адресът на подмрежата му е \texttt{10.16.3.255}.
  \item Мрежата му не е разделена на подмрежи.
  \end{enumerate}
\end{q}

\begin{q}
  Ако хост в мрежа има адрес \texttt{172.16.45.14/30}, какъв е адресът на
  подмрежата, към която принадлежи той?
  \begin{enumerate}
  \item \texttt{172.16.45.0}
  \item \texttt{172.16.45.4}
  \item \texttt{172.16.45.8}
  \item \texttt{172.16.45.12}
  \item \texttt{172.16.45.16}
  \end{enumerate}
\end{q}

\begin{q}
  Коя маска е най-практично да използваме при Point-to-point връзка, за да
  намалим разхода на IP адреси?
  \begin{enumerate}
  \item \texttt{/8}
  \item \texttt{/16}
  \item \texttt{/24}
  \item \texttt{/30}
  \item \texttt{/31}
  \end{enumerate}
\end{q}

\begin{q}
  Кой е адресът на подмрежата на хост с IP адрес \texttt{172.16.66.0/21}?
  \begin{enumerate}
  \item \texttt{172.16.36.0}
  \item \texttt{172.16.48.0}
  \item \texttt{172.16.64.0}
  \item \texttt{172.16.0.0}
  \end{enumerate}
\end{q}

\begin{q}
  На маршрутизатор имате интерфейс с IP адрес \texttt{192.168.192.10/29}. Колко
  хоста могат да имат адреси от локалната мрежа, свързана към интерфейса на
  маршрутизатора? (Маршрутизаторът се брои за хост в подмрежата.)
  \begin{enumerate}
  \item $6$
  \item $8$
  \item $30$
  \item $62$
  \item $126$
  \end{enumerate}
\end{q}

\begin{q}
  Имате нужда да конфигурирате мрежови интерфейс на сървър с IP адрес, който е
  част от подмрежата \texttt{192.168.19.24/29}. На маршрутизатора в тази
  подмрежа е зачислен първият адрес от нея. Кой от следните адреси можете да
  зачислите на сървъра?
  \begin{enumerate}
  \item \texttt{192.168.19.0/255.255.255.0}
  \item \texttt{192.168.19.33/255.255.255.240}
  \item \texttt{192.168.19.26/255.255.255.248}
  \item \texttt{192.168.19.31/255.255.255.248}
  \item \texttt{192.168.19.34/255.255.255.240}
  \end{enumerate}
\end{q}

\begin{q}
  Имате маршрутизатор, свързан с локална мрежа, посредством мрежови интерфейс с
  адрес \texttt{192.168.192.19/29}. Какъв е broadcast адресът, който хостовете в
  подмрежата ще използват?
  \begin{enumerate}
  \item \texttt{192.168.192.15}
  \item \texttt{192.168.192.31}
  \item \texttt{192.168.192.63}
  \item \texttt{192.168.192.127}
  \item \texttt{192.168.192.255}
  \end{enumerate}
\end{q}

\begin{q}
  Имате мрежа, която трябва да разделите на подмрежи, всяка от които да съдържа
  поне 16 хоста. Коя от следните маски бихте използвали, за да постигнете това?
  \begin{enumerate}
  \item \texttt{255.255.255.192}
  \item \texttt{255.255.255.224}
  \item \texttt{255.255.255.240}
  \item \texttt{255.255.255.248}
  \end{enumerate}
\end{q}

\begin{q}
  Пропуснат – извън материала.
\end{q}

\begin{q}
  Ако IP адресът \texttt{172.16.112.1/25} е зачислен на Ethernet порт на
  маршрутизатор, какъв би бил адресът на подмрежата на този порт?
  \begin{enumerate}
  \item \texttt{172.16.112.0}
  \item \texttt{172.16.0.0}
  \item \texttt{172.16.96.0}
  \item \texttt{172.16.255.0}
  \item \texttt{172.16.128.0}
  \end{enumerate}
\end{q}

\begin{q}
  Пропуснат – извън материала.
\end{q}

\begin{q}
  Пропуснат – извън материала.
\end{q}

\begin{q}
  Пропуснат – извън материала.
\end{q}

\begin{q}
  Имате мрежа с подмрежа \texttt{172.16.17.0/22}. Кой от следните адреси е
  валиден адрес на хост от тази подмрежа?
  \begin{enumerate}
  \item \texttt{172.16.17.1/255.255.255.252}
  \item \texttt{172.16.0.1/255.255.240.0}
  \item \texttt{172.16.20.1/255.255.255.254.0}
  \item \texttt{172.16.16.1/255.255.255.240}
  \item \texttt{172.16.18.255/255.255.252.0}
  \item \texttt{172.16.0.1/255.255.255.0}
  \end{enumerate}
\end{q}

\begin{q}
  Порт \texttt{Ethernet0} на маршрутизатора Ви има адрес
  \texttt{172.16.2.1/23}. Кои от следните могат да бъдат валидни адреси на
  хостове, свързани с \texttt{Ethernet0}, посредством локална мрежа? (Изберете
  две.)
  \begin{enumerate}
  \item \texttt{172.16.0.5}
  \item \texttt{172.16.1.100}
  \item \texttt{172.16.1.192}
  \item \texttt{172.16.2.255}
  \item \texttt{172.16.3.0}
  \item \texttt{172.16.3.255}
  \end{enumerate}
\end{q}

\begin{q}
  За да тествате IP стека на локалния си хост, кой от следните адреси бихте
  подали като параметър на командата \texttt{ping}?
  \begin{enumerate}
  \item \texttt{127.0.0.0}
  \item \texttt{1.0.0.127}
  \item \texttt{127.0.0.1}
  \item \texttt{127.0.0.255}
  \item \texttt{255.255.255.255}
  \end{enumerate}
\end{q}
\section{NAT (Network Address Translation)}

\begin{q}
  Кои от следните са недостатъци на използването на NAT? (Изберете две.)
  \begin{defractors}
  \item Спестява публично достъпни IP адреси.
  \item Причинява загуба на end-to-end проследимостта (traceability) на IP.
  \item Увеличава гъвкавостта при свързване с Интернет.
  \item Някои приложения няма да функционират когато мрежовите им връзки
    преминават през NAT.
  \item Намалява случаите на припокриване на IP адреси.
  \item Отразява се негативно върху сигурността на мрежата.
  \item Намалява забавянето при обработка на мрежовия трафик от маршрутизатора.
  \end{defractors}
  \rans б, г
\end{q}

\begin{q}
  Кои от следните са предимства на използването на NAT? (Изберете три.)
  \begin{defractors}
  \item Спестява публично достъпни IP адреси.
  \item Причинява загуба на end-to-end проследимостта на IP.
  \item Увеличава гъвкавостта при свързване с Интернет.
  \item Някои приложения няма да функционират когато мрежовите им връзки
    преминават през NAT.
  \item Намалява случаите на припокриване на IP адреси.
  \item Отразява се негативно върху сигурността на мрежата.
  \item Намалява забавянето при обработка на мрежовия трафик от маршрутизатора.
  \end{defractors}
  \rans а, в, д
\end{q}

\begin{q}
  Кои от следните са видове NAT? (Изберете две.)
  \begin{defractors}
  \item Статичен NAT
  \item IP NAT pool
  \item Двойно превеждане (NAT double-translation)
  \item PAT (Port Address Translation)
  \end{defractors}
  \rans а, г
\end{q}

\begin{q}
  Кои от следните са добри причини за използване на NAT? (Изберете три.)
  \begin{defractors}
    \item Имате нужда да се свържете с Интернет, а хостовете Ви нямат глобално
      уникални IP адреси.
    \item При избор на нов доставчик на Интернет възниква нужда за преномериране
      на цялата Ви мрежа.
    \item Не искате никой хост да има връзка с Интернет.
    \item Искате две вътрешни мрежи с припокриващи се адресни пространства да се
      слеят.
  \end{defractors}
  \rans а, б, г
\end{q}

\begin{q}
  PAT (Port Address Translation) се нарича също:
  \begin{defractors}
    \item Бърз (Fast) NAT
    \item Статичен (Static) NAT
    \item NAT Overload
  \end{defractors}
  \rans в
\end{q}

\section{Статична маршрутизация}

\subsection{Въпроси с отговор в свободен текст}

\begin{q}
  Работите на хост с операционна система GNU/Linux 3.6.10. Напишете командата, с
  която ще въведете запис за мрежа \texttt{172.16.10.0/24} през маршрутизатор
  \texttt{172.16.20.1} в маршрутната таблица на хоста.

  \rans \texttt{route add -net 172.16.10.0/24 gw 172.16.20.1}
\end{q}

\begin{q}
  Хост изпраща пакет към друг хост, намиращ се в отдалечена мрежа. Какви ще са
  MAC адресът и IP адресът на получателя в рамката, която първият хост изпраща
  към зададения му шлюз?
\end{q}

\begin{q}
  Напишете командата, с която като маршрутизатор по подразбиране (default
  router) се задава хостът с адрес \texttt{172.16.40.1}.
\end{q}

\begin{q}
  Пропуснат (извън материала)
\end{q}

\begin{q}
  Пропуснат (излишно сложен)
\end{q}

\begin{q}
  С коя команда се извежда маршрутната таблица?
\end{q}

\begin{q}
  Пропуснат (извън материала)
\end{q}

\begin{q}
  Вярно или грешно: За да установите връзка с отдалечен хост (хост в отдалечена
  мрежа), трябва да знаете MAC адреса на този хост.

  \rans Грешно
\end{q}

\begin{q}
  Вярно или грешно: За да установите връзка с отдалечен хост (хост в отдалечена
  мрежа), трябва да знаете IP адреса на този хост.

  \rans Вярно
\end{q}

\begin{q}
  Пропуснат (извън материала)
\end{q}

\begin{q}
  Намирате се в подходящата командна обвивка на софтуера за маршрутизация Quagga
  v. 0.99.21. С коя команда ще активирате RIP протокола на мрежовия интерфейс
  \texttt{eth2}?
\end{q}

\begin{q}
  Пропуснат (не е показано в клас)
\end{q}

\begin{q}
  Имате маршрутизатор, който разчита на RIPv2 за автоматична конфигурация на
  записите в маршрутната си таблица. При прекъсване на мрежова връзка на
  маршрутизатора, кой механизъм за предотвратяване на маршрутни цикли
  своевременно ще изпрати информация, че пропадналите маршрути са на недостижимо
  разстояние 16?

  \rans Route poisoning
\end{q}

\begin{q}
  Кой механизъм за предотвратяване на маршрутни цикли подтиска изпращането на
  маршрутна информация през интерфейс, по който тя е била получена?

  \rans Split horizon
\end{q}

\begin{q}
  Пропуснат (извън материала)
\end{q}

\subsection{Въпроси с избор на верен отговор}

\begin{q}
  Компанията Eugene ЕАД използва маршрутизатора \texttt{gw1}, за връзка с
  доставчика си на Интернет услуги (ISP). IP адресът на маршрутизатора на
  доставчика е \texttt{206.143.5.2}. Кои от следните команди ще позволят
  установяването на Интернет връзка на цялата мрежа на Еugene ЕАД? (Изберете
  две.)

  \begin{defractors}
  \item \texttt{\# ifconfig eth0 206.154.5.2 netmask 255.255.255.252}
  \item \texttt{\# route add -net 0.0.0.0 netmask 0.0.0.0 gw 206.143.5.2}
  \item \texttt{\# ip route add default via 206.143.5.2}
  \item \texttt{\# route add -net default gw 206.143.5.0}
  \end{defractors}

  \rans б, в
\end{q}

\begin{q}{*}
  В Quagga, коя команда ще предотврати изпращането на RIP съобщения по даден
  интерфейс, но ще остави възможно приемането на съобщения по този интерфейс?

  \begin{defractors}
  \item \texttt{Router(config-if)\#no routing}
  \item \texttt{Router(config-if)\#passive-interface}
  \item \texttt{Router(config-router)\#passive-interface eth0}
  \item \texttt{Router(config-router)\#no routing updates}
  \end{defractors}
\end{q}

\begin{q}
  Кои от твърденията са верни за командата \texttt{route add -net 172.16.4.0
    netmask 255.255.255.0 gw 192.168.4.2}? (Изберете две.)

  \begin{defractors}

  \item Командата се използва за да се установи статичен маршрут.
  \item Използва се метрика по подразбиране.
  \item Командата се използва за създаване на маршрут по подразбиране.
  \item С тази команда се дефинира статичен маршрут към мрежа с адрес
    \texttt{192.168.4.2}
  \end{defractors}

  \rans а, б
\end{q}

\begin{q}
  Пропуснат
\end{q}

\begin{q}
  Пропуснат (извън материала)
\end{q}

\begin{q}
  Кое от следните е най-доброто описание на метода за предотвратяване на
  маршрутни цикли Split Horizon?

  \begin{defractors}

  \item Информацията за маршрут не трябва да бъде изпращана обратно в посоката,
    от която е дошла.
  \item Разделя трафика, когато имаме голяма физическа мрежа.
  \item Задържа редовните обновявания от разпространение по пропаднала връзка.
  \item Не позволява редовните съобщения за обновяване на маршрутната таблица да
    създадат маршрут до недостъпна мрежа.
  \end{defractors}

  \rans а
\end{q}

\begin{q}
  Нека маршрутизаторите Router A, Router B и Router C са свързани
  последователно. Нека хостът Host A е свързан към Router A и хостът Host C е
  свързан към Router C. Кои от следните твърдения ще бъдат верни, ако Host A се
  опитва да комуникира с Host C докато интерфейсът между Router C и Host C е
  деактивиран? (Изберете две.)

  \begin{defractors}
  \item Router C ще използва ICMP, за да информира Host A, че Host C не може да
    бъде достигнат.
  \item Router C ще използва ICMP, за да информира Router B, че Host C не може
    да бъде достигнат.
  \item Router C ще използва ICMP, за да информира Host A, Router A и Router B
    че Host C не може да бъде достигнат.
  \item Router C ще изпрати съобщение от тип „Destination unreachable“.
  \item Router C ще изпрати съобщение за избор на маршрутизатор.
  \end{defractors}

  \rans а, г
\end{q}

\begin{q}
  Кое твърдение е вярно за безкласовите протоколи за маршрутизация (routing
  protocols)? (Изберете две.)

  \begin{defractors}
  \item Не се допуска използването на недопиращи се мрежи.
  \item Позволено е използването на мрежови маски с променлива дължина (VLSM).
  \item RIPv1 е безкласов протокол за маршрутизация.
  \item RIPv2 поддържа безкласова маршрутизация.
  \end{defractors}

  \rans а, г
\end{q}


\end{document}