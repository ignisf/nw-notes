\documentclass{scrartcl}

\usepackage[a4paper]{geometry}

\usepackage{xecyr}
\usepackage{polyglossia}
\setmainlanguage{bulgarian}

\usepackage{libertineotf}

\usepackage{amsmath}
\newtheorem{definition}{Дефиниция}

\usepackage{hyperref}

\begin{document}

\title{Записки}
\subtitle{по компютърни мрежи и комуникации}
\author{\href{mailto:bordjukov@gmail.com}{П. Борджуков}}
\date{}

\maketitle

\section{Увод в TCP/IP}

Протоколите за комуникация съдържат детайли за оформлението на съобщенията и
описват как компютърът реагира, когато получи съобщение или когато възникне
грешка. Те са за информацията, това което са алгоритмите за изчислителните
процеси. Алгоритъмът ни позволява да опишем даден изчислителен процес, без да
има нужда да навлизаме в детайли като това с помощта на кой програмен език ще
бъде реализиран той. По същия начин, \emph{комуникационният} протокол позволява
да бъде установена комуникация между две устройства, без оглед на хардуерното
или софтуерното им обезпечение.

\begin{definition}
  Протокол за комуникация наричаме множеството от правила за синтаксиса,
  семантиката и синхронизацията на комуникацията.
\end{definition}

\end{document}
