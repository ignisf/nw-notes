\documentclass{scrartcl}

\usepackage[a4paper]{geometry}

\usepackage{xecyr}
\usepackage{polyglossia}
\setmainlanguage{bulgarian}

\usepackage{libertineotf}

\usepackage{amsmath}
\newtheorem{definition}{Дефиниция}

\usepackage{hyperref}

\begin{document}

\title{Записки}
\subtitle{по компютърни мрежи и комуникации}
\author{\href{mailto:bordjukov@gmail.com}{П. Борджуков}}
\date{}

\maketitle

\section{Увод в TCP/IP}
Когато мрежите били създадени, компютрите най-често можели да комуникират само с
други компютри от същия производител. В резултат на това, компаниите били
принудени да разгъщат мрежови решения, базирани изцяло на хардуер от една и съща
марка. За да бъде разрешен този проблем, в края на '70-те години бил създаден
\emph{OSI (Open Systems Interconnection) моделът}, чиято цел е да подпомогне
създаването на оперативно съвместими мрежови устройства.

\subsection{Мрежови модели}

\subsubsection{Многослоен подход}
Еталонен модел (reference model) наричаме концептуален план на система, която
осигурява провеждане на комуникация. Той адресира всички процеси, нужни за
ефективна комуникация и разделя тези процеси в логически групи, наречени
\emph{слоеве}. Казваме, че дадена дадена комуникационна система има разслоена
архитектура, когато е конструирана на базата на еталонен модел.

\subsubsection{Предимства на еталонните модели}
Системите, изградени върху еталонен модел наследяват всички предимства на
многослойните архитектури. Някои от тези предимства са:
\begin{itemize}
\item Разграничение на отговорностите – всеки слой е функционално независим от
  останалите. Това позволява разбиване на сложната задача за осигуряване на
  мрежова комуникация на по-прости, независими една от друга задачи.
\item Улесняване на извършването на промени в имплементацията – промяната в един
  слой не засяга останалите слоеве.
\end{itemize}

\subsection{OSI модел}
\begin{definition}
  OSI моделът е \emph{теоретичен} модел на мрежова комуникационна система, който
  организира мрежовата функционалност в седем слоя и определя комуникационните
  интерфейси, както между самите слоеве, така и между крайните точки в мрежата,
  които използват базиран на OSI модела проколен пакет.
\end{definition}

OSI има седем различни слоя, които са разделени на два групи. Горните три слоя
дефинират как приложенията в крайните станции ще комуникират помежду си и с
потребителите.
% TODO да добавя графика с описание на горните три слоя на OSI

Четирите долни слоя определят по какъв начин данните се пренасят през мрежовата
инфраструктура и физическата среда и по какъв начин да бъде реконструиран
потокът от данни, пристигащ от подателя и адресиран към приложение на приемника.
% TODO да добавя графика с описание на долните четири слоя на OSI

\subsection{Слоевете на OSI}
Еталонният OSI модел има следните седем слоя:
\begin{itemize}
\item \emph{Приложен слой (application layer)}\\ Отговаря за човеко-машинния
  интерфейс на мрежовата система – в него се случва интеракцията между системата
  и потребителя ѝ.
  \item Представителен слой (presentation layer)\\ Отговаря преставянето на
    информацията, на базата на определен общ формат. В този слой се извършват
    процесите на шифриране, компресия, превод и т.н.
  \item Сесиен слой (session layer)\\ Този слой е отговорен за създаването,
    управлението и прекратяването на сесии между две комуникиращи програми.
  \item Транспортен слой (transport layer)\\ Транспортният слой предлага услуги,
    които осигуряват сегментиране и реасемблиране на данните идващи от и
    насочени към горните слоеве. Той освобождава сесийния слой от грижата за
    надеждно и ефективно транспортиране на данните между крайните системи
    (оперира със сегменти).
  \item Мрежов слой (network layer)\\ Отговорен за адресацията на участниците в
    мрежата и за откриването на маршрути помежду им (оперира с пакети).
  \item Канален слой (data-link layer)\\ Осигурява доставянето на съобщения до
    правилния участник в мрежата и превежда съобщения от мрежовия слой към
    битове, които физическия слой да предаде (оперира с рамки).
  \item Физически слой (physical layer)\\ Има точно две функции – приемане на
    битове от физическа среда и предаване на битове във физическа среда (оперира
    с битове).
\end{itemize}

\section{TCP/IP}

\begin{definition}
  Протокол за комуникация наричаме множеството от правила за синтаксиса,
  семантиката и синхронизацията на комуникацията.
\end{definition}

Поради сложността на постигане на мрежова комуникация в нехомогенна среда, за
изграждането на мрежова система се налага използването на стекове от протоколи
за комуникация. Най-широко разпространения протоколен стек е \emph{TCP\IP}. Той
следва т.нар. \emph{DoD модел} (кръстен на създателите му – Министерство на
отбраната на САЩ), който е опростена версия на OSI модела. Той е съставен от
четири, вместо от седем слоя:
\begin{itemize}
\item Приложен слой
\item Транспортен слой
\item Интернет слой
\item Слой за мрежова комуникация
\end{itemize}

\end{document}
