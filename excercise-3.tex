\section{NAT (Network Address Translation)}

\begin{q}
  Кои от следните са недостатъци на използването на NAT? (Изберете две.)
  \begin{defractors}
  \item Спестява публично достъпни IP адреси.
  \item Причинява загуба на end-to-end проследимостта (traceability) на IP.
  \item Увеличава гъвкавостта при свързване с Интернет.
  \item Някои приложения няма да функционират когато мрежовите им връзки
    преминават през NAT.
  \item Намалява случаите на припокриване на IP адреси.
  \item Отразява се негативно върху сигурността на мрежата.
  \item Намалява забавянето при обработка на мрежовия трафик от маршрутизатора.
  \end{defractors}
  \rans б, г
\end{q}

\begin{q}
  Кои от следните са предимства на използването на NAT? (Изберете три.)
  \begin{defractors}
  \item Спестява публично достъпни IP адреси.
  \item Причинява загуба на end-to-end проследимостта на IP.
  \item Увеличава гъвкавостта при свързване с Интернет.
  \item Някои приложения няма да функционират когато мрежовите им връзки
    преминават през NAT.
  \item Намалява случаите на припокриване на IP адреси.
  \item Отразява се негативно върху сигурността на мрежата.
  \item Намалява забавянето при обработка на мрежовия трафик от маршрутизатора.
  \end{defractors}
  \rans а, в, д
\end{q}

\begin{q}
  Кои от следните са видове NAT? (Изберете две.)
  \begin{defractors}
  \item Статичен NAT
  \item IP NAT pool
  \item Двойно превеждане (NAT double-translation)
  \item PAT (Port Address Translation)
  \end{defractors}
  \rans а, г
\end{q}

\begin{q}
  Кои от следните са добри причини за използване на NAT? (Изберете три.)
  \begin{defractors}
    \item Имате нужда да се свържете с Интернет, а хостовете Ви нямат глобално
      уникални IP адреси.
    \item При избор на нов доставчик на Интернет възниква нужда за преномериране
      на цялата Ви мрежа.
    \item Не искате никой хост да има връзка с Интернет.
    \item Искате две вътрешни мрежи с припокриващи се адресни пространства да се
      слеят.
  \end{defractors}
  \rans а, б, г
\end{q}

\begin{q}
  PAT (Port Address Translation) се нарича също:
  \begin{defractors}
    \item Бърз (Fast) NAT
    \item Статичен (Static) NAT
    \item NAT Overload
  \end{defractors}
  \rans в
\end{q}
