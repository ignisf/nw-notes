\section{IP адресация}
\begin{q}
  Напишете адреса на подмрежата, broadcast адреса на подмрежата и интервала от
  валидни адреси на хостове за всяка от следните двойки адрес и маска:
  \begin{defractors}
  \item \texttt{192.168.100.25/30}
  \item \texttt{192.168.100.37/255.255.255.240}
  \item \texttt{192.168.100.66/255.255.255.224}
  \item \texttt{192.168.100.17/29}
  \item \texttt{192.168.100.99/26}
  \item \texttt{192.168.100.99/255.255.255.128}
  \end{defractors}
\end{q}

\begin{q}
  Имате клас B мрежа и се нуждаете от 29 подмрежи. Каква мрежова маска ще
  изберете? \rans \texttt{255.255.248.0} или \texttt{/21}
\end{q}

\begin{q}
  Какъв е broadcast адресът на подмрежата, в която се намира хостът с адрес
  \texttt{192.168.192.10/29}? \rans \texttt{192.168.192.15}
\end{q}

\begin{q}
  Колко адреса за хостове предлага подмрежа с маска \texttt{/29}? \rans 6
\end{q}

\begin{q}
  Какъв е адресът на подмрежата на \texttt{10.16.3.65/23}? \rans
  \texttt{10.16.2.0}
\end{q}

\begin{q}
  Попълнете следната таблица:

  \begin{center}
    \begin{tabular}{ r c c } \toprule
      \textbf{\emph{CIDR}} & \textbf{\emph{Маска на подмрежа (dot-decimal)}} &
        \textbf{\emph{Брой хостове в подмрежа}} \\ \midrule
      /16 & & \\ \hline
      /17 & & \\ \hline
      /18 & & \\ \hline
      /19 & & \\ \hline
      /20 & & \\ \hline
      /21 & & \\ \hline
      /22 & & \\ \hline
      /23 & & \\ \hline
      /24 & & \\ \hline
      /25 & & \\ \hline
      /26 & & \\ \hline
      /27 & & \\ \hline
      /28 & & \\ \hline
      /29 & & \\ \hline
      /30 & & \\ \bottomrule
    \end{tabular}
  \end{center}
\end{q}

\begin{q}
  Попълнете следната таблица:

  \begin{center}
    \begin{tabular}{@{}lc p{2cm} p{2cm} c c@{}} \toprule
      & & \multicolumn{2}{c}{Брой битове за} & \multicolumn{2}{c}{Брой} \\
      Адрес & Клас & \centering подмрежа & \centering хост & подмрежи & хостове
      \\ \midrule
      \texttt{10.25.66.154/23} & & & & & \\
      \texttt{172.31.254.12/24} & & & & & \\
      \texttt{192.168.20.123/28} & & & & & \\
      \texttt{63.24.89.21/18} & & & & & \\
      \texttt{128.1.1.254/20} & & & & & \\
      \texttt{208.100.54.209/30} & & & & \\ \bottomrule
    \end{tabular}
  \end{center}
\end{q}

\begin{q}
  Какъв е максималният брой IP адреси, които могат да бъдат зачислени на хостове
  в локална подмрежа с маска \texttt{255.255.255.224}?
  \begin{defractors}
  \item 14
  \item 15
  \item 16
  \item 30
  \item 31
  \item 62
  \end{defractors}
\end{q}

\begin{q}
  Имате мрежа, която трябва да разделите на 29 подмрежи, предлагащи възможно
  най-голям брой адреси на хостове. Колко бита трябва да заемете от полето на
  хоста, за да постигнете това?
  \begin{defractors}
  \item 2
  \item 3
  \item 4
  \item 5
  \item 6
  \item 7
  \end{defractors}
\end{q}

\begin{q}
  Имате хост с IP адрес \texttt{200.10.5.68/28}. Кой е адресът на подмрежата, от
  която е част този хост?
  \begin{defractors}
  \item \texttt{200.10.5.56}
  \item \texttt{200.10.5.32}
  \item \texttt{200.10.5.64}
  \item \texttt{200.10.5.0}
  \end{defractors}
\end{q}

\begin{q}
  Колко подмрежи и колко адреса за хостове в подмрежа предоставя мрежовият адрес
  \texttt{172.16.0.0/19}?
  \begin{defractors}
  \item $7$ подмрежи, $30$ хоста във всяка
  \item $7$ подмрежи, $2046$ хоста във всяка
  \item $7$ подмрежи, $8190$ хоста във всяка
  \item $8$ подмрежи, $30$ хоста във всяка
  \item $8$ подмрежи, $2046$ хоста във всяка
  \item $8$ подмрежи, $8190$ хоста във всяка
  \end{defractors}
\end{q}

\begin{q}
  Кои две твърдения са верни за IP адреса \texttt{10.16.3.65/23}?
  \begin{defractors}
  \item Адресът на подмрежата му е \texttt{10.16.3.0/255.255.254.0}.
  \item Най-ниският адрес на хост в подмрежата му е \texttt{10.16.2.1}.
  \item Последният валиден адрес на хост в подмрежата му е \texttt{10.16.2.254}.
  \item Broadcast адресът на подмрежата му е \texttt{10.16.3.255}.
  \item Мрежата му не е разделена на подмрежи.
  \end{defractors}
\end{q}

\begin{q}
  Ако хост в мрежа има адрес \texttt{172.16.45.14/30}, какъв е адресът на
  подмрежата, към която принадлежи той?
  \begin{defractors}
  \item \texttt{172.16.45.0}
  \item \texttt{172.16.45.4}
  \item \texttt{172.16.45.8}
  \item \texttt{172.16.45.12}
  \item \texttt{172.16.45.16}
  \end{defractors}
\end{q}

\begin{q}
  Коя маска е най-практично да използваме при Point-to-point връзка, за да
  намалим разхода на IP адреси?
  \begin{defractors}
  \item \texttt{/8}
  \item \texttt{/16}
  \item \texttt{/24}
  \item \texttt{/30}
  \item \texttt{/31}
  \end{defractors}
\end{q}

\begin{q}
  Кой е адресът на подмрежата на хост с IP адрес \texttt{172.16.66.0/21}?
  \begin{defractors}
  \item \texttt{172.16.36.0}
  \item \texttt{172.16.48.0}
  \item \texttt{172.16.64.0}
  \item \texttt{172.16.0.0}
  \end{defractors}
\end{q}

\begin{q}
  На маршрутизатор имате интерфейс с IP адрес \texttt{192.168.192.10/29}. Колко
  хоста могат да имат адреси от локалната мрежа, свързана към интерфейса на
  маршрутизатора? (Маршрутизаторът се брои за хост в подмрежата.)
  \begin{defractors}
  \item $6$
  \item $8$
  \item $30$
  \item $62$
  \item $126$
  \end{defractors}
\end{q}

\begin{q}
  Имате нужда да конфигурирате мрежови интерфейс на сървър с IP адрес, който е
  част от подмрежата \texttt{192.168.19.24/29}. На маршрутизатора в тази
  подмрежа е зачислен първият адрес от нея. Кой от следните адреси можете да
  зачислите на сървъра?
  \begin{defractors}
  \item \texttt{192.168.19.0/255.255.255.0}
  \item \texttt{192.168.19.33/255.255.255.240}
  \item \texttt{192.168.19.26/255.255.255.248}
  \item \texttt{192.168.19.31/255.255.255.248}
  \item \texttt{192.168.19.34/255.255.255.240}
  \end{defractors}
\end{q}

\begin{q}
  Имате маршрутизатор, свързан с локална мрежа, посредством мрежови интерфейс с
  адрес \texttt{192.168.192.19/29}. Какъв е broadcast адресът, който хостовете в
  подмрежата ще използват?
  \begin{defractors}
  \item \texttt{192.168.192.15}
  \item \texttt{192.168.192.31}
  \item \texttt{192.168.192.63}
  \item \texttt{192.168.192.127}
  \item \texttt{192.168.192.255}
  \end{defractors}
\end{q}

\begin{q}
  Имате мрежа, която трябва да разделите на подмрежи, всяка от които да съдържа
  поне 16 хоста. Коя от следните маски бихте използвали, за да постигнете това?
  \begin{defractors}
  \item \texttt{255.255.255.192}
  \item \texttt{255.255.255.224}
  \item \texttt{255.255.255.240}
  \item \texttt{255.255.255.248}
  \end{defractors}
\end{q}

\begin{q}
  Пропуснат – извън материала.
\end{q}

\begin{q}
  Ако IP адресът \texttt{172.16.112.1/25} е зачислен на Ethernet порт на
  маршрутизатор, какъв би бил адресът на подмрежата на този порт?
  \begin{defractors}
  \item \texttt{172.16.112.0}
  \item \texttt{172.16.0.0}
  \item \texttt{172.16.96.0}
  \item \texttt{172.16.255.0}
  \item \texttt{172.16.128.0}
  \end{defractors}
\end{q}

\begin{q}
  Пропуснат – извън материала.
\end{q}

\begin{q}
  Пропуснат – извън материала.
\end{q}

\begin{q}
  Пропуснат – извън материала.
\end{q}

\begin{q}
  Имате мрежа с подмрежа \texttt{172.16.17.0/22}. Кой от следните адреси е
  валиден адрес на хост от тази подмрежа?
  \begin{defractors}
  \item \texttt{172.16.17.1/255.255.255.252}
  \item \texttt{172.16.0.1/255.255.240.0}
  \item \texttt{172.16.20.1/255.255.255.254.0}
  \item \texttt{172.16.16.1/255.255.255.240}
  \item \texttt{172.16.18.255/255.255.252.0}
  \item \texttt{172.16.0.1/255.255.255.0}
  \end{defractors}
\end{q}

\begin{q}
  Порт \texttt{Ethernet0} на маршрутизатора Ви има адрес
  \texttt{172.16.2.1/23}. Кои от следните могат да бъдат валидни адреси на
  хостове, свързани с \texttt{Ethernet0}, посредством локална мрежа? (Изберете
  две.)
  \begin{defractors}
  \item \texttt{172.16.0.5}
  \item \texttt{172.16.1.100}
  \item \texttt{172.16.1.192}
  \item \texttt{172.16.2.255}
  \item \texttt{172.16.3.0}
  \item \texttt{172.16.3.255}
  \end{defractors}
\end{q}

\begin{q}
  За да тествате IP стека на локалния си хост, кой от следните адреси бихте
  подали като параметър на командата \texttt{ping}?
  \begin{defractors}
  \item \texttt{127.0.0.0}
  \item \texttt{1.0.0.127}
  \item \texttt{127.0.0.1}
  \item \texttt{127.0.0.255}
  \item \texttt{255.255.255.255}
  \end{defractors}
\end{q}