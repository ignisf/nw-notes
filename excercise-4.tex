\section{Статична маршрутизация}

\subsection{Въпроси с отговор в свободен текст}

\begin{q}
  Работите на хост с операционна система GNU/Linux 3.6.10. Напишете командата, с
  която ще въведете запис за мрежа \texttt{172.16.10.0/24} през маршрутизатор
  \texttt{172.16.20.1} в маршрутната таблица на хоста.

  \rans \texttt{route add -net 172.16.10.0/24 gw 172.16.20.1}
\end{q}

\begin{q}
  Хост изпраща пакет към друг хост, намиращ се в отдалечена мрежа. Какви ще са
  MAC адресът и IP адресът на получателя в рамката, която първият хост изпраща
  към зададения му шлюз?
\end{q}

\begin{q}
  Напишете командата, с която като маршрутизатор по подразбиране (default
  router) се задава хостът с адрес \texttt{172.16.40.1}.
\end{q}

\begin{q}
  Пропуснат (извън материала)
\end{q}

\begin{q}
  Пропуснат (излишно сложен)
\end{q}

\begin{q}
  С коя команда се извежда маршрутната таблица?
\end{q}

\begin{q}
  Пропуснат (извън материала)
\end{q}

\begin{q}
  Вярно или грешно: За да установите връзка с отдалечен хост (хост в отдалечена
  мрежа), трябва да знаете MAC адреса на този хост.

  \rans Грешно
\end{q}

\begin{q}
  Вярно или грешно: За да установите връзка с отдалечен хост (хост в отдалечена
  мрежа), трябва да знаете IP адреса на този хост.

  \rans Вярно
\end{q}

\begin{q}
  Пропуснат (извън материала)
\end{q}

\begin{q}
  Намирате се в подходящата командна обвивка на софтуера за маршрутизация Quagga
  v. 0.99.21. С коя команда ще активирате RIP протокола на мрежовия интерфейс
  \texttt{eth2}?
\end{q}

\begin{q}
  Пропуснат (не е показано в клас)
\end{q}

\begin{q}
  Имате маршрутизатор, който разчита на RIPv2 за автоматична конфигурация на
  записите в маршрутната си таблица. При прекъсване на мрежова връзка на
  маршрутизатора, кой механизъм за предотвратяване на маршрутни цикли
  своевременно ще изпрати информация, че пропадналите маршрути са на недостижимо
  разстояние 16?

  \rans Route poisoning
\end{q}

\begin{q}
  Кой механизъм за предотвратяване на маршрутни цикли подтиска изпращането на
  маршрутна информация през интерфейс, по който тя е била получена?

  \rans Split horizon
\end{q}

\begin{q}
  Пропуснат (извън материала)
\end{q}

\subsection{Въпроси с избор на верен отговор}

\begin{q}
  Компанията Eugene ЕАД използва маршрутизатора \texttt{gw1}, за връзка с
  доставчика си на Интернет услуги (ISP). IP адресът на маршрутизатора на
  доставчика е \texttt{206.143.5.2}. Кои от следните команди ще позволят
  установяването на Интернет връзка на цялата мрежа на Еugene ЕАД? (Изберете
  две.)

  \begin{defractors}
  \item \texttt{\# ifconfig eth0 206.154.5.2 netmask 255.255.255.252}
  \item \texttt{\# route add -net 0.0.0.0 netmask 0.0.0.0 gw 206.143.5.2}
  \item \texttt{\# ip route add default via 206.143.5.2}
  \item \texttt{\# route add -net default gw 206.143.5.0}
  \end{defractors}

  \rans б, в
\end{q}

\begin{q}{*}
  В Quagga, коя команда ще предотврати изпращането на RIP съобщения по даден
  интерфейс, но ще остави възможно приемането на съобщения по този интерфейс?

  \begin{defractors}
  \item \texttt{Router(config-if)\#no routing}
  \item \texttt{Router(config-if)\#passive-interface}
  \item \texttt{Router(config-router)\#passive-interface eth0}
  \item \texttt{Router(config-router)\#no routing updates}
  \end{defractors}
\end{q}

\begin{q}
  Кои от твърденията са верни за командата \texttt{route add -net 172.16.4.0
    netmask 255.255.255.0 gw 192.168.4.2}? (Изберете две.)

  \begin{defractors}

  \item Командата се използва за да се установи статичен маршрут.
  \item Използва се метрика по подразбиране.
  \item Командата се използва за създаване на маршрут по подразбиране.
  \item С тази команда се дефинира статичен маршрут към мрежа с адрес
    \texttt{192.168.4.2}
  \end{defractors}

  \rans а, б
\end{q}

\begin{q}
  Пропуснат
\end{q}

\begin{q}
  Пропуснат (извън материала)
\end{q}

\begin{q}
  Кое от следните е най-доброто описание на метода за предотвратяване на
  маршрутни цикли Split Horizon?

  \begin{defractors}

  \item Информацията за маршрут не трябва да бъде изпращана обратно в посоката,
    от която е дошла.
  \item Разделя трафика, когато имаме голяма физическа мрежа.
  \item Задържа редовните обновявания от разпространение по пропаднала връзка.
  \item Не позволява редовните съобщения за обновяване на маршрутната таблица да
    създадат маршрут до недостъпна мрежа.
  \end{defractors}

  \rans а
\end{q}

\begin{q}
  Нека маршрутизаторите Router A, Router B и Router C са свързани
  последователно. Нека хостът Host A е свързан към Router A и хостът Host C е
  свързан към Router C. Кои от следните твърдения ще бъдат верни, ако Host A се
  опитва да комуникира с Host C докато интерфейсът между Router C и Host C е
  деактивиран? (Изберете две.)

  \begin{defractors}
  \item Router C ще използва ICMP, за да информира Host A, че Host C не може да
    бъде достигнат.
  \item Router C ще използва ICMP, за да информира Router B, че Host C не може
    да бъде достигнат.
  \item Router C ще използва ICMP, за да информира Host A, Router A и Router B
    че Host C не може да бъде достигнат.
  \item Router C ще изпрати съобщение от тип „Destination unreachable“.
  \item Router C ще изпрати съобщение за избор на маршрутизатор.
  \end{defractors}

  \rans а, г
\end{q}

\begin{q}
  Кое твърдение е вярно за безкласовите протоколи за маршрутизация (routing
  protocols)? (Изберете две.)

  \begin{defractors}
  \item Не се допуска използването на недопиращи се мрежи.
  \item Позволено е използването на мрежови маски с променлива дължина (VLSM).
  \item RIPv1 е безкласов протокол за маршрутизация.
  \item RIPv2 поддържа безкласова маршрутизация.
  \end{defractors}

  \rans а, г
\end{q}
