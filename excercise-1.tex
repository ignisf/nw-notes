\section{Увод в TCP/IP}
\subsection{Въпроси относно OSI модела}
\subsubsection{С отговор в свободен текст}
\begin{q}
  Кой слой е отговорен за конвертирането на данните от каналния слой в
  електрически импулси?
\end{q}

\begin{q}
  В кой слой е имплементирано маршрутизирането, позволяващо свързването и
  избирането на път за пренос на данни между две крайни системи?
\end{q}

\begin{q}
  Кой слой определя как се форматират, представят, кодират и конвертират
  мрежовите данни?
\end{q}

\begin{q}
  Кой слой е отговорен за създаването, управляването и прекратяването на сесии
  между приложения?
\end{q}

\begin{q}
  Кой слой осигурява сигурното предаване на данни през физическата среда и
  отговаря основно за физическото адресиране, дисциплината на линията, мрежовата
  топология, нотификацията за грешки, преноса на рамки в правилен ред и контрола
  на потока?
\end{q}

\begin{q}
  Кой слой се използва за надеждна комуникация между крайни хостове в мрежата и
  предоставя механизми за установяване, поддържане и прекратяване на виртуални
  вериги, откриване на и възстановяване от грешки, възникнали при транспорта на
  данни, и контрол на потока на информация?
\end{q}

\begin{q}
  Кой слой предоставя логическо адресиране, което маршрутизаторите използват за
  установяване на маршрут за пренос на данни?
\end{q}

\begin{q}
  Кой слой определя волтажа, скоростта и изводите (pinout) на проводника и
  предава битове между мрежови устройства?
\end{q}

\begin{q}
  Кой слой комбинира битове в байтове и байтове в рамки, използва MAC адресиране
  и установява дали са възникнали грешки по време на преноса на данните във
  физическата среда?
\end{q}

\begin{q}
  Кой слой е отговорен за разграничаването на данните от различните приложения
  (мултиплексиране) при мрежова комуникация?
\end{q}

\begin{q}
  Продукт на кой слой са рамките?
\end{q}

\begin{q}
  Продукт на кой слой са сегментите?
\end{q}

\begin{q}
  Продукт на кой слой са пакетите?
\end{q}

\begin{q}
  Продукт на кой слой са битовете?
\end{q}

\begin{q}
  Поставете следните единици данни в ред на енкапсулация, започвайки от
  най-вътрешната:

  \begin{itemize}
  \item Пакети
  \item Рамки
  \item Битове
  \item Сегменти
  \end{itemize}
\end{q}

\begin{q}
  Кой слой сегментира и реасемблира данните?
\end{q}

\begin{q}
  Кой слой се грижи за привеждането на данните във формат, удобен за предаване
  на физическо ниво и отговаря за нотификацията при възникване на грешки,
  мрежовата топология и контрола на потока?
\end{q}

\begin{q}
  Кой слой управлява адресирането на устройствата, проследява положението на
  устройствата в мрежата и определя най-добрия път за пренос на данни?
\end{q}

\begin{q}
  Каква е дължината в битове и в какъв вид се изразява MAC адресът?
\end{q}

\begin{q}
  Кой слой създава виртуална верига преди да започне да изпраща данни?

  \rans Транспортният
\end{q}

\begin{q}
  Върху кои слоеве е дефиниран Ethernet?

  \rans Канален и физически
\end{q}

\begin{q}
  В кой слой се използва логическото адресиране на хостовете в мрежата?

  \rans Мрежовият
\end{q}

\begin{q}
  В кой слой се дефинират хардуерните адреси на мрежовите интерфейси на
  хостовете?

  \rans Каналният
\end{q}

\subsubsection{С избор на верен отговор}
\begin{q}
  Приемащ хост не е успял да получи всички сегменти, чието пристигане трябва да
  потвърди. Какво може да направи хоста, за да подобри надеждността на
  комуникационната сесия?

  \begin{enumerate}
  \item Да изпрати различен номер на изходящ порт.
  \item Да рестартира виртуалната верига.
  \item Да намали sequence номера.
  \item Да намали големината на прозореца.
  \end{enumerate}

  \rans г
\end{q}

\begin{q}
  Когато станция изпрати съобщение до MAC адреса \texttt{ff:ff:ff:ff:ff:ff},
  към кой вид съобщения може да бъде причислено то?

  \begin{enumerate}
  \item Unicast
  \item Multicast
  \item Anycast
  \item Broadcast
  \end{enumerate}

  \rans г
\end{q}

\begin{q}
  В кой слой се извършва сегментацията на данните?

  \begin{enumerate}
  \item Физически
  \item Канален
  \item Мрежови
  \item Транспортен
  \end{enumerate}

  \rans г
\end{q}

\begin{q}
  Маршрутизаторите оперират на слой №\dots, LAN комутаторите оперират на слой
  №\dots, LAN концентраторите оперират на слой №\dots, текстообработката се
  извършва в слой №\dots

  \begin{enumerate}
  \item 3, 3, 1, 7
  \item 3, 2, 1, никой
  \item 3, 2, 1, 7
  \item 2, 3, 1, 7
  \item 3, 3, 2, никой
  \end{enumerate}

  \rans в
\end{q}

\begin{q}
  Коя е правилната последователност на енкапсулация на данните?

  \begin{enumerate}
  \item Данни, рамка, пакет, сегмент, бит
  \item Сегмент, данни, пакет, рамка, бит
  \item Данни, сегмент, пакет, рамка, бит
  \item Данни, сегмент, рамка, пакет, бит
  \end{enumerate}

  \rans в
\end{q}

\begin{q}
  За кой слой са характерни потвържденията (acknowledgements), последователното
  номериране (sequencing) и контрола на потока?

  \begin{enumerate}
  \item Слой 2
  \item Слой 3
  \item Слой 4
  \item Слой 7
  \end{enumerate}

  \rans в
\end{q}

\subsection{Въпроси относно DoD модела}
\begin{q}
  Как се отнасят слоевете на DoD модела към слоевете на OSI?
\end{q}

\begin{q}
  Идентифицирайте слоя от DoD модела, към който принадлежи всеки един от
  следните протоколи:

  \begin{itemize}
  \item Internet Protocol (IP) \rans Интернет
  \item Telnet \rans Приложен
  \item FTP \rans Приложен
  \item SNMP \rans Приложен
  \item DNS \rans Приложен
  \item Address Resolution Protocol (ARP) \rans Интернет
  \item DHCP/BootP \rans Приложен
  \item Transmission Control Protocol (TCP) \rans Транспортен
  \item User Datagram Protocol (UDP) \rans Транспортен
  \item NFS \rans Приложен
  \item Internet Control Message Protocol (ICMP) \rans Интернет
  \item Reverse Address Resolution Protocol (RARP) \rans Интернет
  \item Proxy ARP \rans Интернет
  \item TFTP \rans Приложен
  \item SMTP \rans Приложен
  \item Ethernet \rans Канален
  \end{itemize}
\end{q}

\begin{q}
  Кой от следните са слоеве на DoD модела? (Изберете три.)

  \begin{enumerate}
  \item Приложен слой
  \item Сесиен слой
  \item Транспортен слой
  \item Интернет слой
  \item Физически слой
  \end{enumerate}
\end{q}

\begin{q}
  Кой слой от DoD модела е еквивалентен на мрежовия слой от OSI модела?

  \begin{enumerate}
  \item Приложен
  \item Транспортен
  \item Интернет
  \item Канален
  \end{enumerate}
\end{q}