%%% Local Variables:
%%% TeX-master: "excercises"
%%% ispell-check-comments: exclusive
%%% ispell-local-dictionary: "bg"
%%% End:

\section{Увод в TCP/IP}
\subsection{Въпроси относно OSI модела}

\begin{q}
  Кой слой е отговорен за конвертирането на данните от каналния слой в
  електрически импулси?
\end{q}

\begin{q}
  В кой слой е имплементирано маршрутизирането, позволяващо свързването и
  избирането на път за пренос на данни между две крайни системи?
\end{q}

\begin{q}
  Кой слой определя как се форматират, представят, кодират и конвертират
  мрежовите данни?
\end{q}

\begin{q}
  Кой слой е отговорен за създаването, управляването и прекратяването на сесии
  между приложения?
\end{q}

\begin{q}
  Кой слой осигурява сигурното предаване на данни през физическата среда и
  отговаря основно за физическото адресиране, дисциплината на линията, мрежовата
  топология, нотификацията за грешки, преноса на рамки в правилен ред и контрола
  на потока?
\end{q}

\begin{q}
  Кой слой се използва за надеждна комуникация между крайни хостове в мрежата и
  предоставя механизми за установяване, поддържане и прекратяване на виртуални
  вериги, откриване на и възстановяване от грешки, възникнали при транспорта на
  данни, и контрол на потока на информация?
\end{q}

\begin{q}
  Кой слой предоставя логическо адресиране, което маршрутизаторите използват за
  установяване на маршрут за пренос на данни?
\end{q}

\begin{q}
  Кой слой определя волтажа, скоростта и изводите (pinout) на проводника и
  предава битове между мрежови устройства?
\end{q}

\begin{q}
  Кой слой комбинира битове в байтове и байтове в рамки, използва MAC адресиране
  и установява дали са възникнали грешки по време на преноса на данните във
  физическата среда?
\end{q}

\begin{q}
  Кой слой е отговорен за разграничаването на данните от различните приложения
  (мултиплексиране) при мрежова комуникация?
\end{q}

\begin{q}
  Продукт на кой слой са рамките?
\end{q}

\begin{q}
  Продукт на кой слой са сегментите?
\end{q}

\begin{q}
  Продукт на кой слой са пакетите?
\end{q}

\begin{q}
  Продукт на кой слой са битовете?
\end{q}

\begin{q}
  Поставете следните единици данни в ред на енкапсулация, започвайки от
  най-вътрешната:

  \begin{itemize}
  \item Пакети
  \item Рамки
  \item Битове
  \item Сегменти
  \end{itemize}
\end{q}

\begin{q}
  Кой слой сегментира и реасемблира данните?
\end{q}

\begin{q}
  Кой слой се грижи за привеждането на данните във формат, удобен за предаване
  на физическо ниво и отговаря за нотификацията при възникване на грешки,
  мрежовата топология и контрола на потока?
\end{q}

\begin{q}
  Кой слой управлява адресирането на устройствата, проследява положението на
  устройствата в мрежата и определя най-добрия път за пренос на данни?
\end{q}

\begin{q}
  Каква е дължината в битове и в какъв вид се изразява MAC адресът?
\end{q}